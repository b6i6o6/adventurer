\documentclass{book}                    % For two-sided book page style
\usepackage[a5paper, marginparwidth=0.5in]{geometry}  % For a5 paper size
\usepackage{palatino}                   % For Palatino fonts
\usepackage{suffix}                     % For starred commands (tlmgr bigfoot)
\usepackage{pdflscape}                  % For landscape map pages
\usepackage{import}                     % For importing nested folders
\usepackage{fancyhdr}                   % For complicated headers and footers
% \usepackage{marginnote}                 % For faction sidenotes
\usepackage{etoolbox}
\usepackage{moresize}
\usepackage{sidenotes}
\usepackage{caption}
\usepackage[space]{grffile}
\usepackage[export]{adjustbox}
\usepackage[utf8]{inputenc}
\usepackage[T1]{fontenc}
% \geometry{bindingoffset=1.5cm}

% Make captions closer to figures for sidenotes quest markers
\setlength{\abovecaptionskip}{0pt}

% Clear default headers and footers
\pagestyle{fancy}
\fancyhead{}
\fancyfoot{}
\renewcommand{\headrulewidth}{0pt}

% \newcommand{\sectionbreak}{\cleardoublepage}
\setlength{\parskip}{\baselineskip}
\newwrite\tempfile

\begin{document}

\newcommand {\race}{adventurer}
\newcommand {\class}{adventurer}
\newcommand {\character}{Adventurer}
\newcommand {\gender}{male}
\newcommand {\localelang}{enUS}
\newcommand {\map}[1]{
    \fancyhead{}
    \fancyfoot{}
    %https://tex.stackexchange.com/questions/155925/how-to-get-included-pdf-pages-adjusted-to-left-and-right-pages-in-a-book
    \mbox{}\par\vspace{6.8cm}
    \hspace*{-.4in}\makebox[\textwidth][c]{\raisebox{-.5\height}[0pt][0pt]{\includegraphics[height=0.6\paperheight,angle=90]{Maps/#1.jpg}}}\par\label{#1}
    \vfill\mbox{}
}
\newcommand {\prepareQuestURI}[1] {
    \immediate\openout\tempfile=page.tmp
    \immediate\write\tempfile{#1}
    \immediate\closeout\tempfile
    \immediate\write18{cat page.tmp | perl uri_escape.pl > page_uri.tmp}
}
\newcommand {\custom}[3][] {
    \prepareQuestURI{#2}
    \input|"./cache.sh #1 | perl extract_data.pl '#3' \character"
}
\newcommand {\quest}[1] {
    \marker{Quest}{0.5}{#1}
    \custom[Quest]{#1}{Description}
}
\newcommand {\questA}[1] {
    \quest{#1 (Alliance)}
}
\newcommand {\questH}[1] {
    \quest{#1 (Horde)}
}
\newcommand {\start}[1] {
    \quest{#1}
}
\newcommand {\startA}[1] {
    \questA{#1}
}
\newcommand {\startH}[1] {
    \questH{#1}
}
\newcommand {\complete}[1] {
    \marker{QuestComplete}{0.5}{#1}
    \custom[Quest]{#1}{Completion}
}
\newcommand {\completeA}[1] {
    \complete{#1 (Alliance)}
}
\newcommand {\completeH}[1] {
    \complete{#1 (Horde)}
}
\newcommand {\progress}[1] {
    \custom[Quest]{#1}{Progress}
}
\newcommand {\progressA}[1] {
    \progress{#1 (Alliance)}
}
\newcommand {\progressH}[1] {
    \progress{#1 (Horde)}
}
\newcommand {\rare}[1]{%
    \leavevmode\marker[0]{Rare}{0.5}{#1}
}
\WithSuffix\newcommand\progress*[1] {
    % TODO: Improve the perl script to *not* do 'npc says' in this case
    STAR\custom[Quest]{#1}{Progress}
}
\newcommand {\dialogue}[2][] {          % For NPCs, can be overriden
    \custom[#1]{#2}{Dialogue}
}
\newcommand {\quotes}[2][Quotes] {      % For NPCs, cannot be overriden
    \custom{#2}{#1}
}
\newcommand {\flower}[1] {
    % TODO: Make an index of flowers
    #1
}
\newcommand {\fetch}[7] {%
    \input|"perl fetch_resource.pl #1 #2 #3 #4 #5 #6 #7"
}
\newcommand {\npc}[1] {%
    \fetch{npc}{#1}{\localelang}{\character}{\race}{\class}{\gender}%
}
\newcommand {\text}[1] {%
    \fetch{broadcast_text}{#1}{\localelang}{\character}{\race}{\class}{\gender}%
}
\newcommand {\speech}[3] {
    \par\npc{#1}#2: \text{#3}
}
\newcommand {\says}[2] {
    \speech{#1}{says}{#2}
}
\newcommand {\whispers}[2] {
    \speech{#1}{whispers}{#2}
}
\newcommand {\yells}[2] {
    \speech{#1}{yells}{#2}
}
\newcommand*\cleartoleftpage{
  \clearpage
  \ifodd\value{page}\hbox{}\newpage\fi
}
\newcommand {\zone}[2] {
    \cleartoleftpage
    \map{#1}
    \chapter*{\centering #1}
    \fancyhead[L]{
        \sc #1
        \includegraphics[scale=0.4,valign=c]{Icons/MinimapArrowSmall.png} \pageref{#1}
    }
    \fancyfoot[LE,RO]{\thepage}
    \fancyfoot[RE,LO]{
        \includegraphics[scale=0.4,valign=m]{Icons/Hearthstone.png} \pageref{\currenthearthstone}
    }
    \addcontentsline{toc}{chapter}{#1}
    \import{"Zones/#1/"}{"#1"}
}
\newcommand {\breadcrumb}[1] {
    \fancyhead[R]{\sc #1}
    \subsection*{\centering#1}
    \addcontentsline{toc}{section}{#1}
    \label{#1}
    \input{"#1"}
}
\newcommand {\subzone}[1] {
    \clearpage
    \breadcrumb{#1}
}
\newcommand {\ask}[1] {
    \noindent$\longrightarrow$ \textit{#1}\newline
}
\newcommand {\goto}[2][] {
    \indent$\hookrightarrow$ #1 \includegraphics[scale=0.4,valign=m]{Icons/Book.png} \textit{\pageref{#2}}
}
% Hack to have margin notes always on the left
\makeatletter
\patchcmd{\@mn@margintest}{\@tempswafalse}{\@tempswatrue}{}{}
\patchcmd{\@mn@margintest}{\@tempswafalse}{\@tempswatrue}{}{}
\reversemarginpar
\makeatother
\newcommand {\marker}[4][0.5]{%
    % \leavevmode\marginnote{
    %     \centering\includegraphics[scale=#2]{Icons/#1.png}
    %     \centering\ssmall{#3}
    % }\ignorespaces
    \begin{marginfigure}[#1cm]%
        \centering
        \includegraphics[scale=#3]{Icons/#2.png}
        \caption*{\centering\ssmall{#4}}%
    \end{marginfigure}%
}
\newcommand {\alliance}[1]{
    \marker{Alliance}{0.7}{}#1\newline
}
\newcommand {\horde}[1]{
    \marker{Horde}{0.7}{}#1\newline
}
\newcommand {\miss}[1]{#1}  % Used to indicate what is missing from wowpedia
\newcommand {\currenthearthstone}{}
\newcommand {\hearthstone}[1] {
    \renewcommand {\currenthearthstone} {#1}
}

\renewcommand{\contentsname}{Flight Master\label{Flight Master}}

\addtocontents{toc}{
    Eager to travel fast around the world, you make your way to the Flight Master. He eyes you warily until you hand him a small pouch of coins. He suddenly looks way more willing to discuss business with you.

    Flight Master says: Where do you want to go, \class ?\par
}

\title{
    World of Warcraft\\\character
}
\author{Blizzard Entertainment}
\date{}
\maketitle

% \part{Cataclysm (1-60)}

\zone{Teldrassil}

\zone{Darkshore}

\zone{Southern Barrens}

\zone{Tanaris}

\zone{Silithus}

\import{Zones/Hearthstone/}{Hearthstone}

\tableofcontents

\end{document}
